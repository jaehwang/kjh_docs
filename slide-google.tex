\begin{frame}[allowframebreaks]
\frametitle{Developing Products at the Speed and Scale of Google}

\begin{quote}
개발자 5,000+명, office 40 군데, 프로젝트 2,000개, 빌드 하루에 50,000번(일년에 18,000,000), 테스트 케이스 1억개
\end{quote}

\begin{itemize}

\item Code consistency: code review, analysis, test, ...
    \begin{itemize}
    \item C++를 많이 사용. 타 언어 혼용, 라이브러리 등 복잡한 의존 관계
      \begin{itemize}
      \item 코딩 스타일과 자동화가 중요함
      \end{itemize}
%    \item IDE에 통합할 예정
    \end{itemize}

\item 동시에 많은 사람들이 check-out할 때 병목 발생 함
    \begin{itemize}
    \item FUSE를 사용해 수정되는 파일만 on-demand로 repository에서 전송 
         \begin{itemize}
         \item 보통 전체 code의 약  10\% 정도만 수정됨
         \end{itemize}
    \item 버전 관리 도구는 Perforce. 자체 빌드 도구 개발
    \end{itemize}
\item C++코드가 많아 컴파일 시간이 많이 걸림
    \begin{itemize}
    \item 이전 바이너리와 새 object파일들을 이용해 새 binary를 생성
    \item 빌드 속도 10배 향상
    \end{itemize}

\item Build machine: 1000 cores, 50+ TB 메모리
    \begin{itemize}
    \item 계산 중심의 작업은 cloud에서 하도록 작업 진행 중
    \item Action cache: 1 PB용량, 7일간 보관
    \end{itemize}

\pagebreak

\item TODO
    \begin{itemize}
    \item IDE integration 
    \item Code  visualization, search
    \item 더 많은 code analysis, 문서화, ...
    \item .... 
    \end{itemize}
\end{itemize}

\REFERENCE{Reference: \url{http://qconsf.com/sf2010/presentation/Developing+Products+at+the+Speed+and+Scale+of+Google}}

\end{frame}

\begin{frame}
\frametitle{Developing Products at the Speed and Scale of Google}

\begin{itemize}

\item 규모: 5000+ 개발자, 40 office, 2000 프로젝트, 하루에 50000 빌드(일년에 1800만 빌드), 1억  테스트 케이스,  

\item C++를 많이 사용. 타 언어 혼용, 라이브러리 등 복잡한 의존 관계
    \begin{itemize}
    \item 코딩 스타일과 자동화가 중요함
    \end{itemize}

\item 계산 중심의 작업은 cloud에서 하도록 작업 진행 중

\item 동시에 많은 사람들이 check-out할 때 병목 발생 함
    \begin{itemize}
    \item Check-out 시 FUSE를 사용해 수정하는 10%만 writable 하게 하고 나머지는 read-only로 빌드 시 전송 한다고 함
    \item 버전 관리 도구는 Perforce
    \end{itemize}

\item 자체 빌드 도구를 사용한다고 함
    \begin{itemize}
    \item GCC 를 수정해서  이전 바이너리와 새 object파일들을 이용해 새 binary를 만드는 방식을 이용함. 이를 통해 빌드 속도 10배 향상
    \end{itemize}

\item Code consistency: code review, analysis, test, ...
    \begin{itemize}
    \item IDE에 통합할 예정
    \end{itemize}

\item Build machine: 1000 cores, 50+ TB 메모리
    \begin{itemize}
    \item C++코드가 많아 컴파일 시간이 많이 걸림.
    \item Action cache: 1 PB용량, 7일간 보관
    \end{itemize}

\item TODO
    \begin{itemize}
    \item IDE integration 
    \item Code  visualization, search
    \item 더 많은 code analysis, 문서화, ...
    \item .... 
    \end{itemize}
\end{itemize}

\end{frame}

\begin{frame}
\frametitle{Keep Agile Agile}

\begin{itemize}
\item Agile은 risk를 관리하기 위한 것이다.
\item Risk는 fear로 부터 기인한다.
\item Fear는 뭔가를 알기 때문에 생기는 것(rational fear)과 몰라서 생기는 
것(irrational fear)이 있다.
    \begin{itemize}
    \item Rational fear는 좋은 것.
    \item Irrational fear에서 빠져나오는 것을 배우자. 이것이 Agile.
    \end{itemize}
\item BP보다는 ``evolving process based on patterns''이 더 중요하단다.
    BP라는 이름로 정형화 된 절차만을 강요하는 것보다는 context에서 
    유연하게 대처할 수있도록 하는 것이 필요하다는 뜻.
\item Practice와 관련해서 ``Dreyfus model of skill acquisition''을 이용함.
    \begin{enumerate}
    \item Novice: needs practices
    \item Advanced beginner: uses practices
    \item Competent: defines practices
    \item Proficient: falls back on practices
    \item Expert: subverts practices.
    \end{enumerate}
\end{itemize}
\end{frame}

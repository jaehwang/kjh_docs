\begin{frame}[allowframebreaks]
\frametitle{Continuous Dlivery}

\begin{quote}
Software always production ready!
\end{quote}

\begin{itemize}
  \item Prerequisites 
  \begin{itemize}
  \item Excellent automated testing at all levels
  \item Comprehensive configuration management
  \item Continuous integration
  %  \begin{itemize}
  %  \item main stream에 코드를 자주 모으자.
  %  \item 다른 말로는, 자동으로 매일 merge를 하자.
  %  \item Build, unit test, analysis, packaging
  %  \end{itemize}
  \end{itemize}

  \item ``Deployment Pipeline''
  \begin{itemize}
  \item Commit state
  \item Automated acceptance testing
  \item Automated capacity testing
  \item Manual testing
  \item Release
  \end{itemize}
\end{itemize}

\pagebreak

주요 이슈
 
\begin{itemize}
\item Test Automation
  \begin{itemize}
  \item Functional acceptance test, unit test, integration test,
    system test, performance test, security test는 당연히(?) 자동으로
  \item Showcase, exploratory testing, usability testing은 자동화 하기 어려움.
    숙제.
  \end{itemize}
\item Deployment Automation
%: Deployment의 자동화는 주로 네트웍(웹)
%  기반의 서비스 개발에서 그 중요성이 강조된다고 생각된다.
  \begin{itemize}
  \item Feature toggle, blue-green deployments
  \item Canary release, dark launching
  \item 잘못된 deployment의 rollback
  \item DB migration %DBdeploy와 Puppet이라는 도구 선전도 함.
  \end{itemize}
\end{itemize}

\end{frame}

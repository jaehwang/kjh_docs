\begin{frame}
\frametitle{Continuous Dliverty 1/2}
 
Three Prerequisites of Continuous Delivery:
\begin{itemize}
\item Excellent automated testing at all levels
\item Comprehensive configuration management
\item Continuous Integration: feature branch로 인한 문제 해결이 목표
  \begin{itemize}
  \item main stream에 코드를 자주 모으자.
  \item 다른 말로는, 자동으로 매일 merge를 하자.
  \item Build, unit test, analysis, packaging
  \end{itemize}
\end{itemize}
\end{frame}

\begin{frame}
\frametitle{Continuous Dliverty 2/2}

Acceptance test의 자동화와 production environment에 자동으로
deployment하는 것이 핵심이 듯.
 
\begin{itemize}
\item Test automation
  \begin{itemize}
  \item Functional acceptance test, unit test, integration test,
    system test, performance test, security test는 당연히(?) 자동으로
    하는 걸로 보더군.
  \item Showcase, exploratory testing, usability testing은 자동화 하기
    어렵다는 군. 이동네의 숙제인듯.
  \end{itemize}
\item Deployment automation: Deployment의 자동화는 주로 네트웍(웹)
  기반의 서비스 개발에서 그 중요성이 강조된다고 생각된다.
  \begin{itemize}
  \item DBdeploy와 Puppet이라는 도구 선전도 함.
  \item Feature toggle, blue-green deployments와 같은 아이디어도 소개
  \item 잘못된 deployment의 rollback, database migration
  \end{itemize}
\end{itemize}

\end{frame}

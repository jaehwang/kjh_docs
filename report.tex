\documentclass[a4paper]{article}

\usepackage{fullpage}
\usepackage{kotex}

\usepackage{ifpdf} 
\ifpdf
    \usepackage[unicode]{hyperref}
    \input glyphtounicode\pdfgentounicode=1 
\else
    \usepackage[unicode,dvipdfm]{hyperref} 
\fi

\usepackage{pgf}

\def\doframeit#1{\vbox{%
  \hrule height\fboxrule
    \hbox{%
      \vrule width\fboxrule \kern\fboxsep
      \vbox{\kern\fboxsep #1\kern\fboxsep }%
      \kern\fboxsep \vrule width\fboxrule }%
    \hrule height\fboxrule }}

\def\frameit{\smallskip \advance \linewidth by -50.5pt \setbox0=\vbox
  \bgroup \strut \ignorespaces }

\def\endframeit{\ifhmode \par \nointerlineskip \fi \egroup
  \doframeit{\box0}}

\newenvironment{Frame} {\vspace{3mm}\par\begin{frameit}\vspace{-5mm}}
  {\end{frameit}\vspace{-3mm}}

\newcommand{\mbtt}[1]{\texttt{#1}}
\newcommand{\mbit}[1]{\textit{#1}}
\newcommand{\mbrm}[1]{\textrm{#1}}

\newcommand{\REFERENCE}[1]{\vfill\mbox{\tiny #1}}


\title{QCon San Francisco 2010}
\author{김재황 // 소프트웨어센터}

\begin{document}

\maketitle

\section{개요}

\section{세션들}

\subsection{Tutorial 11/1 REST from Scratch}

\begin{figure}[t]
    \begin{Frame}
        \begin{center}
        \pgfimage[width=.8\textwidth,page=1]{slide-rest}
        \end{center}
    \end{Frame}
    \caption{REST from Scratch}
    \label{REST}
\end{figure}

개발자들이 파이선에서 루비로 옮겨 가고 있다는데... 당연히 일은
재미있어야 하지만 자기만 재미있으면 안 될 것이다. 이 아저씨 회사가 하는
일이 교육이라서 신 기술을 좋아하는 건 아니겠지?

\subsection{Tutorial 11/2 Continuous Delivery}

Martin Fowler, Jez Humble, Tom Sulston 세 명이 합동 진행한 튜토리얼.
셋이서 각본을 잘 준비한 건 아닌 것 같았다.

이 분들의 머리 속에는 여러가지 덕목이 기본으로 깔려있겠지만 그 중 다음
이야기를 하나 건졌다.

\begin{quote}
``If it hurts, do it more often, and bring the pain forward.''  
\end{quote}

\begin{figure}[t]
    \begin{Frame}
        \begin{center}
        \pgfimage[width=.8\textwidth,page=1]{slide-continuous-delivery}
        \end{center}
    \end{Frame}
    \caption{Continuous Delivery 1/2}
    \label{Continuous Delivery 1}
\end{figure}

\begin{figure}[t]
    \begin{Frame}
        \begin{center}
        \pgfimage[width=.8\textwidth,page=2]{slide-continuous-delivery}
        \end{center}
    \end{Frame}
    \caption{Continuous Delivery 2/2}
    \label{Continuous Delivery 2}
\end{figure}

그외에 영어가 짧아서 제대로 이해를 못한 건지 아저씨들이 설명을 안한고
넘어간 건지 모를 용어들: canary releasing, dark launching, event
sourcing 등.

\subsection{Opening Keynote 11/3 Innovation at Google}
 
발표자는 Patrick Copeland.
 
이노베이션은 아이디어가 아니라 사람에 의해서 일어나는 것이라고 함.
에디슨의 ``천재는 1\%의 영감과 99\%의 땀으로 만들어진다''는 말을
인용했는데 매우 공감이 감.
 
Pretotyping이라는 concept을 얘기함.
\begin{quote}
Build the right `it' before you build `it' right.
http://pretotyping.org  
\end{quote}
 
들을 때는 다 공감이 갔었는데 이렇게 정리하고 나니 right thing이 idea랑
같은 게 아닌가하는 생각이 드네...

\subsection{Conference 11/3 Clojure-Java Interop: A better Java than Java}

\begin{figure}[t]
    \begin{Frame}
        \begin{center}
        \pgfimage[width=.8\textwidth,page=1]{slide-clojure-java}
        \end{center}
    \end{Frame}
    \caption{Clojure Java Interop}
    \label{clojure java}
\end{figure}

<Programming Clojure>의 저자 Stuart Halloway가 발표자.
 
Clojure가 JVM에서 돌아가는 언어로서는 Java 보다 낫다고 주장. Java 보다
나중에 나왔으니 당연히 그래야죠.
 
\subsection{Parade 11/3 San Francisco Giants의 월드 시리즈 우승 기념 행진}
 
공식적으로 스케줄이 조정이 되어 퍼레이드를 볼 수 있는 40분의 시간이
주어졌다.  퍼레이드는 Market st를 지나가는데 QCon이 진행되고 있는
Westin호텔이 바로 그 길에 있다.
 
거리는 아침부터 인파로 가득찬 상태이지만 나도 (Lotte) Giants 팬인데 안
나갈 수 있나.
 
앞사람 머리만 실컷 보고 말았지만 잠시나마 Giants의 팬으로서 이곳
사람들과 기쁨을 공유했던 일은 Lotte Giants가 우승하기 전까지는 잊지
못할 것 같다.
 
\subsection{Conference 11/3 The Role of Hypermedia and the Future of
  Web Integration}
 
하루종일 했던 튜토리얼을 한 시간으로 압축한 것에 지나지 않음. 다른 데
들어갈 걸 ㅠㅠ.
 
튜토리얼에서 언급하지 않았던 키워드들:
\begin{itemize}
\item mikyung: rule engine in Resfulie. 클라이언트를 좀 쉽게 만들 수
  있도록 한 DSL인듯.
  \begin{itemize}
  \item 부인이 한국인이라더니 부인 이름이 미경인듯. 기회가 있으면 한 번
    물어 봐야지.
  \end{itemize}
\item client++, JSONP, code on demand
\item atom, xhtml, websocket, opensearch, ...
  \begin{itemize}
  \item 튜토리얼 때 조금 설명한 것들도 있지만 오늘 시간이 없는
    와중에는는 언급만 하더군.
  \end{itemize}
\end{itemize}
 
\subsection{Conference 11/3 Developing Products at the Speed and Scale
  of Google}
 
발표자는 Ashish Kumar. Google의 Engineering Tools 팀의 매니저. 팀원이
100명 정도라는 군.  구글의 개발자가 5000명이 넘고, 진행되는 과제는
2000개 정도임. 개발자 50명 당 engineering tool 인원이 1명인 셈.
 
원래 다른 주제 발표가 잡혀있었는데 발표자 집안 일로 변경됨. 이 것이
내게는 행운이 될 줄이야.
 
Engineering tools팀이 하는 일이 LCD TV 연구소에서 했던 SVN, CIS 인프라
구축과 매우 흡사해서 듣는 내내 흥미로웠음.
 
팀이 기본적으로 하는 일은 ``아이디어를 제품으로 만드는 과정을 가속시키는
것''이라고 함. 꽤 마음에 드는 이야기라고 생각됨.

\begin{figure}[t]
    \begin{Frame}
        \begin{center}
        \pgfimage[width=.8\textwidth,page=1]{slide-google}
        \end{center}
    \end{Frame}
    \caption{Deveoping Products at the Speed and Scale of Google}
    \label{google}
\end{figure}
 
이야기를 다 듣고 나니 결국 우리와 비슷한 일을 하지만 차원이 좀 다르다는
느낌을 받았다. 우리는 check out이 느리면 속도가 빠른 소트리지를 사는
정도인데 구글은 파일시스템을 바꿔 버리고, 그것 때문에 빌드가 곤란해지면
그 문제를 해결하는 빌드 툴을 만들었다. 빌드에 시간이 많이 걸리면
컴파일러를 개선한다. 그래도 성에 차지 않으면 하드웨어를 증설한다. 그
하드웨어도 cache, cloud 기술을 이용해 알뜰하게 사용한다. 그리고
구글에서 테스트를 손으로 하고 있으면 원시인 취급을 받을 것이다.

\subsection{Conference 11/3 Keeping Agile Agile}
 
Dan North의 발표.  

매 발표가 끝날 때 마다 초록/노랑/빨강 종이를 한 장씩 항아리에 넣어
발표에 대한 평가를 한다.  오늘 내가 참석한 다른 발표는 다 초록을
넣었는데 이 발표만 노랑을 넣었다.  내용이 없어서. 농담 따먹기는 엄청했다. 
문서를 안 쓰기 시작하면 Agile을 시작하는 거라는 등의 별로 재미 없는 농담.
 
그래도 조금 정리하자면...

\begin{center}
fear -> risk -> process -> hate(meetings, gantt chart, ...)
\end{center}

\begin{figure}[t]
    \begin{Frame}
        \begin{center}
        \pgfimage[width=.8\textwidth,page=1]{slide-agile-agile}
        \end{center}
    \end{Frame}
    \caption{Keep Agile Agiel}
    \label{agile agile}
\end{figure}

인터넷을 좀 검색하니 오늘 talk의 주요 내용을 다룬 기사가 있네요.
http://www.infoq.com/articles/better-best-practices
 
기사를 읽어 보니...

BP는 조직의 expert나 proficient level의 사람들 발목을 잡을 수도 있다.
따라서 BP는 prescriptive 한 것 보다는 descriptive한 것이 좋다는 얘기를
하고자 한듯.
 
\subsection{Conference 11/3 Dataflow Programming: a Scalable
  Data-Centric Approach to Parallelism}
 
발표자는 Jim Falgout.
 
Java로 dataflow programming을 할 수 있는 DataRush라는 솔루션을 설명함.
Multi-core+I/O 또는 multi-node환경에 적용할 수 있다고 함.
 
Dataflow는 DAG 의 edge는 데이터 queue, 노드는 operator로 표현됨.
Erlang이나 Scala의 actor패턴과 유사하다고 함.
 
open cloud consortium의 벤치 마킹 결과 DataRush가 Hadoop보다 성능이 좋다고 함.
 
대용량 데이터를 처리하는 금융업계에서 needs가 많다고 함. 
 
Embedded system에서 multi-core + java를 도입하고 있으므로 이런
환경에서도 활용할 수있으면 좋겠다는 이야기를 의견을 전달했다.
 
\subsection{Ending Keynote 11/3 Forty Years of Fun with Computers}
 
Smalltalk 동네에서 유명한 Dan Ingalls의 무대.
최근에는  Lively Kernel프로젝트를 하신다니 놀랍다.
 
맥에서 Squeak Smalltalk이라는 걸 이용해 자신이 처음 만든 포트란 코드도
보여주시고, 재미있는 Smalltalk프로그램도 직접 실행을 해 주시고,
Lively도 시연을 해주심.
 
발표 자료 자체도 Squeak 에서 실행되는 나로서는 알수 없는 프로그램으로
진행함.
 
관록이 엄청 발산되는 무대였음. 
 
\subsection{Keynote 11/4 Being Elastic - Evolving Programming for the Cloud}
 
eBay의 Randy Shoup의 키노트.
 
Throughput은 리소스와 커뮤니케이션에 의해 결정된다는 Neil Gunter의
”Universal Law of Computational Scalability”이론을 인용함.
 
주요 고려 사항을 하나씩 설명함.
 
\begin{itemize}
\item Prallelism
\item Layers
\item Services
\item Stateless instances. Durable state in persistent storage
\item Key-value data model
\item Failure handling
\item Testing. Automated testing is essential.
\item Configuration injection
\item Instrumentation. Debugger 사용이 어렵기 때문에
\end{itemize}
 
\subsection{Conference 11/4 The Problem with the Browser}
 
Colling Jackson, websec.sv.cms.edu
 
사내 용으로 몇 가지 웹 시스템을 만들면서 겪은 문제 세가지를 꼽으라면 다음 
세가지:
\begin{itemize}
\item 브라우저에 따라 동작 여부가 다른 경우
\item 한글이 제대로 안 보이는 경우
\item MySQL 성능이 저하되는 경우
\end{itemize}
 
트랙은 웹 보안이었지만 제목이 브라우저 문제라 뭔가 기대를 했는데 결과는 완전 꽝.
 
\begin{itemize}
\item 인증을 어떻게 할 것인가: URL 기반, 쿠키 기반, HTTP auth, ...
\item cross site request forgery 대응책
    \begin{itemize}
    \item secret hidden value: rails
    \item referer validation: facebook
    \item custom header: gmail
    \end{itemize}
\item Network  attack은 https로 
\item SSL rebinding
\item browserscope.org 
\end{itemize}
 
기대와는 다른 내용. 내가 잘 못 들어간 거죠.
 
\subsection{Conference 11/4 When is Garbage Not Garage?}
 
Ari Zilka, Terracotta.
 
Ehcache를 오픈소스 버전과 enterprise 버전으로 만드는 테라코타사의 제품 설명회.
 
Ehcache를 이용한 caching으로 성능 개선을 할 수 있는데 다가
BigMemory라는 것을 덧 붙여서 JVM의 gc도 안 일어나도록 함으로써 더 높은
성능 향상을 기대할 수 있단다.
 
BigMemory가 디스크 까지 동원해서 caching 된 object를 붙들고 있음으로써
gc의 overhead를 최소화하는 방식으로 이해된다.
 
\subsection{Conference 11/4 Cliff Jumping for Amateurs and Other
  Illuminating Stories}
 
Scrum 코치로 활동하고 있는 Mike Sutton의 감동(?) 스토리.
 
\subsubsection{챕터 1. Courage under Fire}
 
완전히 의욕을 상실한 팀이 있었다. 깐깐한 CTO가 제시한 모종의 solution을
만들어야 한다.  그런데 의욕을 상실했으니 일이 될 리가 없다. Sutton은
팀원 스스로 문제에 대해 토론하고 해결책을 모색하는 미팅을 반나절 동안
진행하도록 했다. 그 결과 그들 스스로 그럴듯한 해결책을 얻게되었다. 그날
저녁 CTO는 개발팀에게 맥주를 샀다.
 
그리고 프로젝트는 원래보다 빠방한 기능을 예정된 기간보다 조금 앞당겨 마칠 수 
있었다.
 
\subsubsection{챕터 2. Cliff Jumping}
 
미국 금융 회사의 튀니지 사무실에 8명 정도의 개발자들이 모였다. 그들은
모두 처음 만난 사이다.  이 팀을 Sutton이 처음 만났을 때 그들은 Sutton을
매우 따뜻하고 편하게 대해 줬다.
 
좋은 사람들이 모였으니 기대를 했겠지? 하지만 첫 번째 sprint가 끝났을 때
그 팀이 이룬 것은 아무것도 없었다. 좀 더 관찰을 해 보니 이 사람들은
서로에 대한 신뢰가 없었다. Sutton은 서로의 이력에 대해 이야기하는
자리도 만들고, 저녁에 회식 자리도 만들었다. 이슬람 국가라 술은 못
마시고 포커, 축구 시청 등의 활동을 하면서 친목을 도모했다.
 
조금씩 나아지고 있었다.
 
어느날 한 팀원이 Sutton에게 해안의 멋진 절벽으로 가서 cliff jumping을
해보자고 제안을 했다. 처음에 Sutton은 싫다고 했다. 위험하다고 생각했기
때문이다.  하지만 라디오에서 들은 ``역경을 함께 헤쳐 나간 사람들''에 대한
이야기가 생각이나 마음을 바꾸기로 했다.
 
팀원들과 함께 찾아간 절벽은 높이가 25미터 정도 되었다. 벼랑 끝에 서서
오만 생각이 다 들었다고 한다. 하지만 두려움을 떨치고 바다로
뛰어들었다.
 
그날 이후 팀웍은 더 욱 단단해 지고.... 일은 잘 끝났다고 한다.
 
\subsection{Conference 11/4 Hot Swapping Your Engine at 30,000 Feet}
 
Rod Barlow, shopzilla.com
 
꽤 젊어 보이는 친구가 말은 어찌나 잠오게 하는지...
 
shopzilla와 bizrate라는 두 쇼핑 사이트를 각각 1999년과 2004년에 오픈한
후 이용자가 늘면서 페이지 하나 보는데 7초가 걸릴 지경이 됨.
 
운영하는 사이트를 조금씩 수정하기로 함. 이 것이 Agile Spirit이라는 군.
 
Flight Plan: a service-oriented site architecture (resilient to
sub-system failures), and a RESTful layer on top of our search
infrastructure
 
\begin{enumerate}
\item email gone wild
    \begin{itemize}
    \item 사이트가 엉망이 되고 나니 엄청난 메일이 왔다는 소리인듯. 
    \item shopzilla를 redesign해서 architecture를 제대로 잡아보기로 함
    \end{itemize}
\item kitchen sinks don’t scale
    \begin{itemize}
    \item J2EE에서 벗어나기로 함. J2EE가 개수대 구멍이었다는 얘기인듯.
    \item 모든 일은 concurrenty하게 하고, sub system장애 시에도 큰 문제가 없는 architecture를 설계함
    \end{itemize}
\item sharing ain't caring
    \begin{itemize}
    \item shopzilla의 설계를 bizrate에도 적용하기로 함.
    \item 비슷한 사이트이고 code 도 공유하기 때문에 이런 결정을 내렸으나 코드 관리에 상당한 애를 먹음
    \end{itemize}
\item SEO(search engine optimizer) suicide
    \begin{itemize}
    \item Agile정신에 입각해 한 페이지 씩 수정된 버전을 릴리즈하기로 함.
    \item Value와 risk 모두를 조기에 확인하기 위함.
    \item 마지막 6개월은 초각을 다투며 미친 듯이 일함. Agile이 원래 인런가? 질문해 볼 걸하는 생각이 지금 드네.
    \item 페이지가 하나씩 바뀌는 바람에  검색엔진이 괴로웠을 거라는 얘기인듯.
    \end{itemize}
\end{enumerate}
 
\subsection{Conference 11/4 Consistency Models in New Generation Databases}
 
MongoDB를 만드는 10gen.com의 Roger Bodamer의 발표.
 
NoSQL(MongoDB) 계열의 DB를 master와 복수의 slave로 구축하는 경우에도
나름 data consistency를 유지하는 방법이 있다는 말씀.
 
\begin{itemize}
\item Eventually consistent
\item ex) DNS, memcaced, async replication, ...
\item RYOW: read your own write
\end{itemize}
 
다른 NoSQL 계통의 DB도 될 것 같은데 MongoDB는 sharding이라는 기능이
있어 shared key를 하나 지정하면 data를 여러 서버에 나눠 저장할 수
있다고 함.
 
\subsection{Conference 11/4 QCon Camp - Unit Test}
 
저녁에 구글 리셉션이 있었다. 받기가 귀찮게 여겨지는 기념품과 미국인들은
이따위 음식만 먹고 살리가 없다는 생각이 들게 하는 부페가 제공되었다.
 
구글 리셉션 후에는 QCon 참가자들이 자유롭게 조직한 토픽과 일정에 따라
모임을 갖는 시간인 QCon Camp가 열렸다. 8개 정도의 모임이 만들어 졌는데
나는 unit test 모임에 참여 하기로 했다.
 
개인적으로 unit test를 제대로 해 본적이 없어 미국 개발자들은 어떤
생각을 갖고 있는 지, 어떻게 unit test를 하는 지 궁금했다.
LG전자 얘기도 좀 하면서 의견도 듣고 싶고.
 
결론적으로 한 시간 동안 진행된 미팅 동안 한 마디도 하지 못했다. 쪽
팔려서. 차마 내 입으로 ``LG전자 SW 개발자들은 자기가 만든 코드를 테스트
하지 않습니다.''라고 할 수가 없었다. 그렇다고 거짓말도 못하겠고.
 
한 10명 정도 모여 둘러 앉았는데 다들 평소에 고민하고 실행한 경험에서
우러나오는 얘기들을 쏟아 냈다.
 
모인 사람들이 로 Java 개발자들이었는지 unit test의 unit은 하나의
클래스에 대응 되는 것으로 이야기들을 했다. 하나의 unit을 테스트 할 때
필요한 다른 클래스들은 모두 mokc-up으로 만 들어야 한다는 얘기, unit
test는 설계를 검증하는 거라는 얘기 등.
(그런데 이게 어디 책같은 데 나오는 얘긴가?)
 
어쨋거나 이야기는 
\begin{itemize}
\item function test, integration test, acceptance test로 뻗어 나가는 가 싶더니 
\item JDBC를 쓰는데 SQL parser는  mock-up으로 해야 하느냐 아니냐 하기도 하고 
\item mock-up framework은 뭐가 좋다 둥 
\item 우리는 자바 스크립트나 html도 unit test를 한다는 둥
\item 테스트를 자동으로 돌리기 위해 Amazon의 EC2 cloud 이용한다면서 
\end{itemize}
 
시간 제한이 없었다면 밤샐 기세로 이야기를 쏟아냈다. 이러는 동안에 나는
우리 개발자들은 왜 자기 코드에 대한 테스트를 하지 않을까라는 생각에
빠지게 되었다. 나름 내린 결론은 LG전자는 HW를 만드는 기업에서 출발했기
때문이다라는 것이다.
 
HW를 시험하는 데 익숙했던 방식을 아직 버리지 못했다기 보다 SW를
시험하는데 필요한 것들이 아직 채워지지 않았다는 뜻이다. 이렇게 된
이유는 SW를 시험하는데 필요한 것이 비어있다는 것을 아직 우리 스스로가
몸으로 느끼지 않기 때문이 아닐까?
 
SW 가 우리의 육안에 보이도록 해야만 SW에 필요한 시험도 채워질 것 같다.
 
\subsection{Keynote 11/5 Software Design in the 21st Century}
 
Martin Fowler의 키노트.
 
21세기가 시작된지도 10년이 넘었지만(2000년부터 21세기라면) ``21세기의
디자인''은 아직 오지 않은 세상의 디자인에 대한 이야기일 거라는 생각을
들게 했다.
 
세가지 토픽을 다뤘는데 다음과 같다.
 
\subsubsection{DSL(Domain Specific Language)}
 
우기가 익숙한 DSL 예로는 SQL, CSS, HQL, Regex, ... 등이 있다. DSL은 생산성을 향상시킨다.

\begin{figure}[t]
    \begin{Frame}
        \begin{center}
        \pgfimage[width=.8\textwidth,page=1]{slide-dsl}
        \end{center}
    \end{Frame}
    \caption{Dmain Specific Language}
    \label{dsl}
\end{figure}
 
어떤 기능을 state machine으로 정의할 수 있고, state machine을 DSL로
정의하면 매우 쉽게 기능을 구현할 수 있다는 예를 듦.
 
<Domain Specific Languages>라는 Fowler가 쓴 책 선전도 함.
 
소프트웨어를 잘 만들기 위해서는 complexity를 abstraction이라는 도구로
제어를 할 수 있어야 한다는 SICP의 이야기 처럼 DSL이 좋은 소프트웨어를
만드는 한 방법이 분명히 될 수 있을 것이다. 그리고 앞에서 언급했지만
우리 선배들은 매크로라는 것을 이미 우리들에게 물려 주었다.
 
구더기 무서워서 장 못담그냐고 할 수 있지만 한 마디 덪붙이자. DSL은 분명
어떤 부분의 complexity를 낮춰 주지만 DSL 자체는 (complexity가 높을 수도
있고 아닐 수도 있지만) 관리를 해야할 대상이 된다고 생각한다.
 
\subsubsection{Continuous Integration \& Delivery}
 
튜토리얼 때 했던 이야기를 간단하게 요약함. 
 
Fowler의 integration은 main trunk에 소스를 모으는 것을 의미한다는 점을
다시 확인했다.  물론 text merge는 도구가 다 해주니 문제가 아니고,
semantic merge가 제대로 되었는 지를 확인하기 위해 build, test,
analysis 등을 해야 한단다.
 
그리고 integration 즉, merge는 최소한 하루에 한 번은 해야 하는 것이
자기의 규칙이라고 함.
 
\subsubsection{Event Sourcing}
 
Fowler의 홈페이지에 소개되어 있음.
http://martinfowler.com/eaaDev/EventSourcing.html
 
키노트 때 노트에 기록도 하고 대강 이해했다고 생각을 했다. 막상 여기에
정리하려다 보니 Event Sourcing이 뭐에 도움이 되는 지가 명확하지 않아
구글링을 좀 했다.
 
결과는 좀 거시기 한데 Event Sourcing과 같이 나타나는 단어로
CQRS(command and query repsponsibility segregation), DDD(domain-driven
design)가 등장했다. 이중 CQRS는 Fowler가 언급하기도 한 것이다. 이런
류의 단어에 대한 선입견 같은 것이 좀 있어서 마음이 거시기 한
듯. 궁금하신 분들은 직접 찾아 보시길...
 
Event Sourcing은 내가 이해한 바로는 database의 로그와 같은 것을
어플리케이션에 장착하자는 거다. 어플리케이션의 상태 변경을 event들의
sequence로 저장하자는 것이다. 이렇게 하면 나중에 database의 redo, undo
같은 것을 어플리케이션에 대해서도 할 수 있다는 것이다.
 
구글링의 결과지만... Event Sourcing은 enterprise application
architecture를 위한 디자인 패턴이이라고 한다. 따라서 앞의 로그 파일
같은 거라는 얘기는 정확한 것은 아니다.
 
이게 왜 좋으냐? audit trail을 할 수 있어서 좋단다. 이게 처음으로 언급된
좋은 점이어서 꽤 놀랐다. 그다음은 상태 변경을 재현 할 수 있어
debugging에 도움이 된다고 한다.  임의의 event들을 만들면 시험에도 쓸 수
있을 것으로 생각된다.

\subsection{Conference 11/5 Lessons Learned to Lessons Productized at
  Microsoft Developer Division}
 
Tim Wagner, Visual Studio Platform Dev Manager.
 
Visual Studio 2010(VS2010)을 만들면서 있었던 이런 저런 이야기를
함. LG전자도 하나의 제품을 만든 다음에 이런 발표를 한 번씩 하면 어떨까
하는 생각이 듦.
 
\begin{itemize}
\item Population diversity
    \begin{itemize}
    \item 자기 개밥을 먹어봐야한다. 근데 다른 사람도 먹여야 한다. 가능하면 많이.
    \item VS2010은 4000명 정도에게 먹였다고 함.
    \end{itemize}
\item Big rock \& agile
    \begin{itemize}
    \item 개발자가 3000명 정도 됨. 잘못 들은 것 같지 않음.
    \item 전체 portfolio 관리와 기능 별 소규모 팀의 agile 개발 사이의 조화
        \begin{itemize}
        \item unit test, path analysis
        \item detect code ``repeat'' \& fix suggestion
        \item mocking frameworks
        \item statistical analysis of bugs and bug fixes
        \end{itemize}
    \end{itemize}
\item 제품 별 branch에서 큰 기능 단위의 branch로 바꿈.  기능 별 팀 구조라서 그런 듯.
\item Internal code motion dashboard 운영
    \begin{itemize}
    \item 결함, 빌드, 릴리즈 등 정보 파악
    \end{itemize}
\item Customer feedback
    \begin{itemize}
    \item Watson: functionality
    \item PerfWatsons: responsiveness
    \item Email
    \item SQM(software quality management)
    \end{itemize}
\item Architecture
\item Testing
    \begin{itemize}
    \item Memory analysis over time
    \item Memory leak
    \item GC problem
    \end{itemize}
\end{itemize}
 
\subsection{Conference 11/5 Inter-Disciplinary Design in a Large
  Organization Context}
 
Robert Sedor, Lead Architect, Pitney Bowes Engineering
 
편지와 봉투를 인쇄해서 배달해 주는 서비스를 제공하는 솔루션을 만드는
프로젝트를 어떻게 이끌고 있는 지를 이야기 함.
 
우리 말로 하면 기구, 회로, 소프트웨어 개발자가 모여서 하는
과제. 소프트웨어 개발자도 embedded software, web software, desktop
software 등 다양하게 구성되어 있다고 함.
 
한 마디로 요약하면 ``commnunication을 잘 해야 이런 과제를 할 수 있다.''임.
 
그 외에 몇 가지 언급한 사항은 다음과 같다.
\begin{itemize}
\item 설계는 commnication이 쉽도록 만들어야 한다.
    \begin{itemize}
    \item 확장성이 있도록 설계해야 한다.
    \item MVC 모델이 협업에 좋더라
    \end{itemize}
\item Building \& Versioning 을 잘 해야 한다.
    \begin{itemize}
    \item 모든 것을 자동화 해야 한다.
    \item 테스트 자동화에 Squish라는 도구를 사용함
    \end{itemize}
\item 당장하는 과제와 상관이 없더라도 개발자가 관심을 갖는 분야에 시간을 투자할 수 있도록 해야 한다.
\end{itemize}
 
\subsection{Conference 11/5 Perception and Action: an Introduction to
  Clojure's Time Model}
 
Stuart Halloway.
 
이번 QCon 발표자 중 가장 명쾌한 스타일의 발표를 한 사람으로
생각됨. <Programming Clojure>의 저자이기도 한데 이 책을 사고 싶어
졌음.
 
Functional programming의 특징 중 하나가 value-oriented라는
것이다. Imperative language(C, Java, ...)의 variable은 value를 assign
하면 의미하는 value가 바뀐다. Functional programming은 변하지 않는
value를 다룬다. 어떻게 다루냐 하면, value에 function을 적용해서 새로운
value를 만들어 내는 방식으로 다룬다.
 
Clojure가 다루는 데이터도 모두  immutable value이다. 
 
데이터가 변하지 않는 다는 것은 concurrenct programming을 매우 쉽게 할
수 있는 조건이 된다.
 
Clojure 웹 사이트의 다음 두 글을 한 번 보는 것도 좋겠다.  On state and
identity, Concurrent programming
 
\subsection{Conference 11/5 Abstractions at Scale, Our Experience at Twitter}
 
Marius Eriksen. 
 
결론 부터 이야기 하자면 아무 내용이 없는 발표였다. 유일하게 red card를
준 세션.
 
Twitter, Google 이런 곳과 관련된 세션은 항상 빈자리가 없었다. 이번
세션도 마찬가지 여서 난 맨 끝자리에 겨우 앉을 수 있었다. 서서 듣는
사람, 늦게 와서 그냥 돌아간 사람 등도 있었고.
 
정말 충격적으로 한시간 내내 Scala, database, memory, CPU, GC, VM,
MapReduce, Google App Engine, Rails, Django, BigTable, LWT, .... 이런
것들의 개요만 나열했다.
 
20대로 보이던데 자기가 아는 걸 보여 주고 싶은 생각에 빠져 있어서
였을까? 자기가 한 일을 설명하면 더 좋았을 텐데...
 
중간에 빠져 나가는 사람들이 속출했는데 난 혹시나 하고 계속 있었다. 
 
\subsection{Conference 11/5 Many-Core Java}
 
Alex Buckley, Java Platform Group at Oracle Corporation
 
Oracle 직원이 Java를 소개하러 나온 모습을 보니 뭔가 어색한 기분이
들었다. Oracle이 사악해지고 있지 않나 하는 걱정 때문이었을까?
 
Scala, Clojure 그 외에 내가 모르는 뭔가들이 concurrency에 적합한 언어로
사람들 사이에 인식되어 가고 있는 상황에서 Oracle도 가만 있지는 않았던
것이다.

2009년 말부터 Lambda 프로젝트를 시작했다고 한다. 2012년으로 예정된
Java SE8에 정식 릴리즈 하는 것이 목표라고 한다.
http://openjdk.java.net/projects/lambda/
 
Filter, sort, map, reduce 등 functional programming에서 cocurrent
programming의 주요 idiom으로 자리 잡아 가고 있는 함수들을 Java도
제공하겠다는 것이다. 그래서 아래와 같은 Java code를 만들 수 있게
하겠다고 했다.
 
hi\_score = students.filter(\#\{Student s -> s.graduate == 2010\}).map(\#\{s -> s.socre\}).max();
 
보다 시피 \#\{ arg -> code\} 형태의 closure를 값으로 쓸수 있게 하고,
filter/map 등은 병렬화 하겠다는 것이 핵심이다. 푸른색 부분을 보면 type
inference도 해 준다는 것을 알 수 있다.
 
여기까지는 좋았다. 하지만 역시 Java는 ”Kingdom of Noun”의 왕이었다.
 
Lambda abstraction을 쓰려면 이른바 SAM(single abstract method)
인터페이스를 정의해야 한단다.  SAM의 예로는 Runnable, Callable,
Comparator, ActionListener 등이 있다. Closure를 쓰려면 type 별로
메소드가 하나인 인터페이스를 정의해야 하는 것이다.
 
여기서부터 사람들이 술렁거렸다. Buckley는 수백만 개발자를 거느린
Java로서는 기존 코드와의 호환성을 고려하지 않을 수가 없단다. 나로서는
매우 아쉽다. Oracle도 심사 숙고해서 내린 결정일라 믿는 수 밖에. Java가
마음에 안 들면 Scala나 Clojure를 쓰면 되고.

\section{정리}

2010년 11월 6일 새별 2시 8분.
샌프란시스코에서

\end{document}

\documentclass[a4paper]{article}

\usepackage{fullpage}
\usepackage{kotex}

\usepackage{ifpdf} 
\ifpdf
    \usepackage[unicode]{hyperref}
    \input glyphtounicode\pdfgentounicode=1 
\else
    \usepackage[unicode,dvipdfm]{hyperref} 
\fi

\usepackage{pgf}

\def\doframeit#1{\vbox{%
  \hrule height\fboxrule
    \hbox{%
      \vrule width\fboxrule \kern\fboxsep
      \vbox{\kern\fboxsep #1\kern\fboxsep }%
      \kern\fboxsep \vrule width\fboxrule }%
    \hrule height\fboxrule }}

\def\frameit{\smallskip \advance \linewidth by -50.5pt \setbox0=\vbox
  \bgroup \strut \ignorespaces }

\def\endframeit{\ifhmode \par \nointerlineskip \fi \egroup
  \doframeit{\box0}}

\newenvironment{Frame} {\vspace{3mm}\par\begin{frameit}\vspace{-5mm}}
  {\end{frameit}\vspace{-3mm}}

\newcommand{\mbtt}[1]{\texttt{#1}}
\newcommand{\mbit}[1]{\textit{#1}}
\newcommand{\mbrm}[1]{\textrm{#1}}

\newcommand{\REFERENCE}[1]{\vfill\mbox{\tiny #1}}


\title{QCon San Francisco 2010}
\author{김재황}

\begin{document}

\maketitle

\section{개요}

\section{세션들}

\subsection{REST}

\begin{figure}[t]
    \begin{Frame}
        \begin{center}
        \pgfimage[width=.8\textwidth,page=1]{slide-rest}
        \end{center}
    \end{Frame}
    \caption{REST from Scratch}
    \label{REST}
\end{figure}

개발자들이 파이선에서 루비로 옮겨 가고 있다는데... 당연히 일은
재미있어야 하지만 자기만 재미있으면 안 될 것이다. 이 아저씨 회사가 하는
일이 교육이라서 신 기술을 좋아하는 건 아니겠지?

\subsection{Continuous Delivery}

\begin{figure}
\pgfimage[width=.8\textwidth,page=1]{slide-continuous-delivery}
\end{figure}


\end{document}

\begin{frame}[fragile]
\frametitle{REST from Scratch}

\begin{center}
REST = Representational State Transfer
\end{center}

\begin{itemize}
\item 고유한 식별자(URI)를 갖는 resource 존재
\item Resource의 representation을 표준 interface(e.g. HTTP)로 통신
\item 제약 조건
\begin{itemize}
\item Client-server
\item Stateless
\item Cacheable
\item Layered system
\item Code on demand
\item Uniform interface
\end{itemize}
\item REST style architecture의 예: WWW
\end{itemize}

\end{frame}

\begin{frame}[fragile,allowframebreaks]
\frametitle{Restfulie Example}

\begin{itemize}
\item Restfulie: \url{http://restfulie.caelum.com.br/}
\[
\mbox{REST} + \mbox{HATEOAS} \Rightarrow \mbox{Restfulie framework}
\]
\item HATEOAS: Hypermedia as the Engine of Application State
\end{itemize}

Example Resources:
\begin{itemize}
    \item trip: \mbtt{http://TRIP/f/523a3f23}
    \item payment: URI unknown at compile time
\end{itemize}

\begin{block}<1->{Trip}
\lstset{basicstyle=\ttfamily\small,language=xml}
\begin{lstlisting}
<flight>
 <information> <from>Seoul</from> <to>Busan</to> 
 </information>
 <value>100</value>
 <link rel="payment" href="http://PAY/a/23342q4"/>
</flight>
\end{lstlisting}
\end{block}

\pagebreak

TRIP과 별도 사이트인 PAY를 통해 비용 지불.

\begin{block}<1->{Code example: payment}
\lstset{basicstyle=\ttfamily\small,language=java}
\begin{lstlisting}
flight = Client.at('http://TRIP/f/523a3f23').get();
confirmation = 
    flight.getLink("payment").patch(cardInfo, value);
\end{lstlisting}
\end{block}

Calender에 여행 일정 추가 후, calendar에서 일정 삭제하면 TRIP과 PAY에서도 
취소 처리가 되도록 할 수도 있다.

\begin{block}<1->{Code example: calendar integration}
\lstset{basicstyle=\ttfamily\small,language=java}
\begin{lstlisting}
me = Client.at('http://CAL').with(auth).get();
me.link("calendar").patch(flight.link("self"));
\end{lstlisting}
\end{block}

\end{frame}

% \begin{frame}
% \frametitle{REST from Scratch}
% 
% Guilherme Silveira, Caelum.
%  
% REST = Representational State Transfer
% 
% \begin{itemize}
% \item 시공을 통합하는 어플리케이션을 만들면 좋겠네.
%   \begin{itemize}
%   \item google/apple apps: 지도, 캘린더, ...
%   \item 리치 클라이언트: 자바스크립트를 실행 할 수 있는. 보안 문제를
%     해결해야
%   \item URI망?
%   \end{itemize}
% \item  WEB, HTML, XHTML, microformats, json, xforms, ... 뭐시 너무 많다.
% \item DSL: comlexity를 숨기기 위한 방법. 유지보수는 어떻게?
% \item Versioning: 서버 기술도 급격히 계속 발전하고, 브라우저도
%   다양하고, 브라우저도 계속 개선되니... 버전 관리 방안도 중요
% \end{itemize}
% 
% \end{frame}

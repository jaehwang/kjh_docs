\documentclass[10pt,unicode,serif,compress,slidetop]{beamer}

\usepackage{amsmath,amssymb,amsthm}
\usepackage{listings}
\def\doframeit#1{\vbox{%
  \hrule height\fboxrule
    \hbox{%
      \vrule width\fboxrule \kern\fboxsep
      \vbox{\kern\fboxsep #1\kern\fboxsep }%
      \kern\fboxsep \vrule width\fboxrule }%
    \hrule height\fboxrule }}

\def\frameit{\smallskip \advance \linewidth by -50.5pt \setbox0=\vbox
  \bgroup \strut \ignorespaces }

\def\endframeit{\ifhmode \par \nointerlineskip \fi \egroup
  \doframeit{\box0}}

\newenvironment{Frame} {\vspace{3mm}\par\begin{frameit}\vspace{-5mm}}
  {\end{frameit}\vspace{-3mm}}


\usepackage{kotex}

\usepackage{pgf}

\usetheme[secheader]{Boadilla}
\usecolortheme{lily}
\usecolortheme{orchid}

\usepackage{apacite}
\bibliographystyle{apa}

\title{QCon San Francisco 2010}
\author{김재황}
\institute{소프트웨어 센터}

\begin{document}

\begin{frame}[plain]
    \titlepage
\end{frame}

\section{QCon}
\begin{frame}
\frametitle{QCon San Francisco, 2010}

\begin{quote}
``QCon is a practitioner-driven conference designed for team leads, architects and project management.''

- \url{http://qconsf.com/sf2010/}
\end{quote}

\begin{itemize}
  \item Tutorial(11월1일 - 2일)
  \item Conference(11월3일 - 5일)
\end{itemize}

\begin{quote}
Agile, Java, NoSQL, REST, 개발과 운영, architecture, cloud, 보안, parallel programming, ...
\end{quote}

\end{frame}


\begin{frame}
\frametitle{주의}

본 문서의 내용 중에는 저자가 잘 못 듣거나 잘 못 이해한 부분이 
포함되어 있을 수 있습니다.

\end{frame}

\section{Development Infrastructure}
\subsection{Continuous Delivery}
\begin{frame}[allowframebreaks]
\frametitle{Continuous Dlivery}

\begin{quote}
Software always production ready!
\end{quote}

\begin{itemize}
  \item Prerequisites 
  \begin{itemize}
  \item Excellent automated testing at all levels
  \item Comprehensive configuration management
  \item Continuous integration
  %  \begin{itemize}
  %  \item main stream에 코드를 자주 모으자.
  %  \item 다른 말로는, 자동으로 매일 merge를 하자.
  %  \item Build, unit test, analysis, packaging
  %  \end{itemize}
  \end{itemize}

  \item ``Deployment Pipeline''
  \begin{itemize}
  \item Commit state
  \item Automated acceptance testing
  \item Automated capacity testing
  \item Manual testing
  \item Release
  \end{itemize}
\end{itemize}

\pagebreak

주요 이슈
 
\begin{itemize}
\item Test Automation
  \begin{itemize}
  \item Functional acceptance test, unit test, integration test,
    system test, performance test, security test는 당연히(?) 자동으로
  \item Showcase, exploratory testing, usability testing은 자동화 하기 어려움.
    숙제.
  \end{itemize}
\item Deployment Automation
%: Deployment의 자동화는 주로 네트웍(웹)
%  기반의 서비스 개발에서 그 중요성이 강조된다고 생각된다.
  \begin{itemize}
  \item Feature toggle, blue-green deployments
  \item Canary release, dark launching
  \item 잘못된 deployment의 rollback
  \item DB migration %DBdeploy와 Puppet이라는 도구 선전도 함.
  \end{itemize}
\end{itemize}

\end{frame}

\subsection{Engineering Tools Team of Google}
\begin{frame}[allowframebreaks]
\frametitle{Developing Products at the Speed and Scale of Google}

\begin{quote}
개발자 5,000+명, office 40 군데, 프로젝트 2,000개, 빌드 하루에 50,000번(일년에 18,000,000), 테스트 케이스 1억개
\end{quote}

\begin{itemize}

\item Code consistency: code review, analysis, test, ...
    \begin{itemize}
    \item C++를 많이 사용. 타 언어 혼용, 라이브러리 등 복잡한 의존 관계
      \begin{itemize}
      \item 코딩 스타일과 자동화가 중요함
      \end{itemize}
%    \item IDE에 통합할 예정
    \end{itemize}

\item 동시에 많은 사람들이 check-out할 때 병목 발생 함
    \begin{itemize}
    \item FUSE를 사용해 수정되는 파일만 on-demand로 repository에서 전송 
         \begin{itemize}
         \item 보통 전체 code의 약  10\% 정도만 수정됨
         \end{itemize}
    \item 버전 관리 도구는 Perforce. 자체 빌드 도구 개발
    \end{itemize}
\item C++코드가 많아 컴파일 시간이 많이 걸림
    \begin{itemize}
    \item 이전 바이너리와 새 object파일들을 이용해 새 binary를 생성
    \item 빌드 속도 10배 향상
    \end{itemize}

\item Build machine: 1000 cores, 50+ TB 메모리
    \begin{itemize}
    \item 계산 중심의 작업은 cloud에서 하도록 작업 진행 중
    \item Action cache: 1 PB용량, 7일간 보관
    \end{itemize}

\pagebreak

\item TODO
    \begin{itemize}
    \item IDE integration 
    \item Code  visualization, search
    \item 더 많은 code analysis, 문서화, ...
    \item .... 
    \end{itemize}
\end{itemize}

\REFERENCE{Reference: \url{http://qconsf.com/sf2010/presentation/Developing+Products+at+the+Speed+and+Scale+of+Google}}

\end{frame}

\subsubsection{Cascade}
\begin{frame}
\frametitle{FUSE(Filesystem in USErspace)}

\begin{itemize}
\item 완전한 filesystem을 userspace에서 구현
\item Linux kernel 2.4, 2.6에서 실행
\end{itemize}

\begin{center}
\pgfimage[width=.6\textwidth]{fuse_structure.png}
\end{center}

\REFERENCE{\url{http://fuse.sourceforge.net/}}

\end{frame}

\begin{frame}[shrink]
\frametitle{FUSE+Subversion}

Example: Conifer Systems사의 Cascade

\begin{center}
\pgfimage[width=.6\textwidth]{cfs.png}
\end{center}

\begin{itemize}
\item Cache
  \begin{itemize}
  \item Repository 데이터 caching
  \item Working directory를 cascade filesystem으로 mount
  \item Cache를 복수의 working directory에서 공유
  \end{itemize}
\item FUSE 기반의 Cascade Filesystem
  \begin{itemize}
  \item Working directory를 cascade filesystem으로 mount
  \item Cache를 복수의 working directory에서 공유
  \end{itemize}
\item Operation
  \begin{itemize}
  \item Working directory에서 파일 내용을 읽거나 바꿀 때만 file data 전송 
  \item \mbtt{svn}을 대체하는 \mbtt{csc} command line interface 제공
  \end{itemize}
\end{itemize}

\REFERENCE{\url{http://www.conifersystems.com/cascade/}}

\end{frame}

\subsection{Cloud Computing}
\begin{frame}
\frametitle{Being Elastic - Evolving Programming for the Cloud}

\begin{itemize}
\item Prallelism
\item Layers
\item Services
\item Stateless instances. Durable state in persistent storage
\item Key-value data model
\item Failure handling
\item Testing. Automated testing is essential.
\item Configuration injection
\item Instrumentation. ($\because$ Debugger 사용이 어렵다.)
\end{itemize}

\end{frame}


\section{Java and Concurrency}

\subsection{Lambda}
\begin{frame}[fragile]
\frametitle{Many-Core Java}

Project Lambda: 
\begin{itemize}
\item 2009년 12월부터 시작. 2012년 Java SE8에 릴리즈 하는 것이 목표.
\item \mbtt{http://openjdk.java.net/projects/lambda/}
\end{itemize}

\lstset{language=Java,basicstyle=\ttfamily\small}
\begin{block}<+->{Example}
\begin{lstlisting}
hiscore = 
    students.filter(#{Student s -> s.graduate == 2010})
            .map   (#{s -> s.socre}) 
            .max   ();
\end{lstlisting}
\end{block}

\begin{columns}[t]
    \begin{column}{0.5\textwidth}
        Ideas
        \begin{itemize}
        \item Closure
        \item SAM(single abstract method)
        \item Type Inference
        \end{itemize}
    \end{column}

    \begin{column}{0.5\textwidth}
        But
        \begin{itemize}
        \item Library
        \item Kingdom of Noun
        \end{itemize}
    \end{column}
\end{columns}

\end{frame}


\subsection{Clojure}
\begin{frame}
\frametitle{Clojure-Java Interop: A better Java than Java}

\begin{itemize}
\item Simplicity: functional programming language가 일반적으로 succinct 하죠.
  \begin{itemize}
  \item 놀라운 것은 Lisp 기반의 언어가 괄호 때문에 이상해 보인다는
    주장이 옳지 않다는 증거를 제시함. 실제로 괄호 수가 Java보다
    Clojure가 적더구만.
  \end{itemize}

\item Immutable data $\rightarrow$ uncontrolled mutation이 없음
  \begin{itemize}
  \item 이건 정말 좋은 점. Concurrent programming에 유리한 조건이 됨.
  \end{itemize}

\item Direct access to JVM: unboxed math
  \begin{itemize}
  \item 숫자 같은 primitive type data를 object 가 아닌 primitive
    type으로 바로 사용할 수 있음. (v1.3 이후)
  \end{itemize}

\item Meta data: 임의의 annotation을 할 수 있음.
  \begin{itemize}
  \item Javadoc하고는 비교가 안 되는 멋진 기능이라는데... 뭔지 정확히
    몰라서...
  \end{itemize}
\end{itemize}

\end{frame}

\begin{frame}[fragile]
\frametitle{Concurrent Programming in Clojure}

\begin{center}
Identity $\stackrel{?}{=}$ State
\end{center}

\begin{itemize}
\item YES: Object Oriented Languages(Java, C++, Python, Ruby, ...)
  \begin{itemize}
  \item Identity와 identity의 state가 구분 안 됨
  \item Object(identity)는 메모리에 저장된 state(value)
  \item Lock이나 protocol이 없으면 stable state 보장 못 함
  \end{itemize}
\item NO: Clojure
  \begin{itemize}
  \item 현재 state는 identity와 관련된 value
  \item Identity는 시간에 따라 다른 value(state)를 가질 수 있음
    \begin{itemize}
    \item 특정 시점에 identity가 가지는 state의 snapshot으로 충분?
    \end{itemize}
  \item Lock이 필요 없음
    \begin{itemize}
    \item Programming language/system에서 처리함
    \item Software transactional memory, \mbtt{ref}, \mbtt{alter}, \mbtt{dosync}, ...
    \end{itemize}
  \end{itemize}
\end{itemize}

\vfill

\mbox{\tiny Reference: \mbtt{http://clojure.org/concurrent\_programming}}

\end{frame}


\begin{frame}[fragile,allowframebreaks=1]
\frametitle{Concurrent Programming in Clojure: Example}
\lstdefinelanguage{clojure}
    {morekeywords={import,defn,let,map,ref,replicate,fn,dotimes,dosync,doseq,
                   alter,range}
    }
\lstset{numbers=left,numberstyle=\tiny,stepnumber=1,numbersep=3pt,basicstyle=\ttfamily\small,language=clojure,keywordstyle=\color{blue}}
\begin{block}<+->{Software transactional memory 예제}
\begin{lstlisting}
(import '(java.util.concurrent Executors))
(defn test-stm [nitems nthreads niters]
  (let [refs  (map ref (replicate nitems 0))
        pool  (Executors/newFixedThreadPool nthreads)
        tasks (map (fn [t]
                      (fn []
                        (dotimes [n niters]
                          (dosync
                            (doseq [r refs]
                              (alter r + 1 t))))))
                   (range nthreads))]
    (doseq [future (.invokeAll pool tasks)]
      (.get future))
    (.shutdown pool)
    (map deref refs)))
(test-stm 2 2 2)
\end{lstlisting}
\end{block}

\pagebreak

\mbtt{(test-stm 2 2 2)}실행하면 리스트 \mbtt{((ref 0) (ref 0))}을 2개의 
thread가 각각 2번씩 업데이트하게된다. 실행은 다음과 같다:

$a$와 $b$가 각각 $0$이라면 $c$와 $d$는 각각 $6$이된다.

\[
\begin{array}{ll}
a & b
\end{array}
\]

\[
\mbrm{execution}\left[
\begin{array}{ll}
  \mbrm{thr 0}\left\{
  \begin{array}{l}
    \mbrm{iter 0}\left(
    \begin{array}{ll}
    +1 & +1 \\
    +0 & +0 
    \end{array}
    \right.\\[2ex]
    \mbrm{iter 1}\left(
    \begin{array}{ll}
    +1 & +1 \\
    +0 & +0 
    \end{array}
    \right.
  \end{array}
  \right.
& 
  \mbrm{thr 1}\left\{
  \begin{array}{l}
    \mbrm{iter 0}\left(
    \begin{array}{ll}
    +1 & +1 \\
    +1 & +1 
    \end{array}
    \right.\\[2ex]
    \mbrm{iter 1}\left(
    \begin{array}{ll}
    +1 & +1 \\
    +1 & +1 
    \end{array}
    \right.
  \end{array}
  \right.
\end{array}
\right]
\]

\[
\begin{array}{ll}
c & d
\end{array}
\]

Identity? State? Value?

\end{frame}


\section{Culture}
\subsection{Pretotyping}
\begin{frame}
\frametitle{Innovation at Google}

The Pretotyping Manifesto

\begin{itemize}
    \item innovators beat ideas 
    \item pretotypes beat productypes 
    \item data beats opinions 
    \item doing beats talking 
    \item simple beats complex 
    \item now beats later 
    \item commitment beats committees
\end{itemize}

\texttt{http://pretotyping.org}

\end{frame}

\subsection{Agile and BP}
\begin{frame}
\frametitle{Keep Agile Agile}

\begin{itemize}
\item Agile은 risk를 관리하기 위한 것이다.
\item Risk는 fear로 부터 기인한다.
\item Fear는 뭔가를 알기 때문에 생기는 것(rational fear)과 몰라서 생기는 
것(irrational fear)이 있다.
    \begin{itemize}
    \item Rational fear는 좋은 것.
    \item Irrational fear에서 빠져나오는 것을 배우자. 이것이 Agile.
    \end{itemize}
\item BP보다는 ``evolving process based on patterns''이 더 중요하단다.
    BP라는 이름로 정형화 된 절차만을 강요하는 것보다는 context에서 
    유연하게 대처할 수있도록 하는 것이 필요하다는 뜻.
\item Practice와 관련해서 ``Dreyfus model of skill acquisition''을 이용함.
    \begin{enumerate}
    \item Novice: needs practices
    \item Advanced beginner: uses practices
    \item Competent: defines practices
    \item Proficient: falls back out of practices
    \item Expert: doesn't know practices.
    \end{enumerate}
\end{itemize}
\end{frame}


\section{REST and DSL}
\subsection{REST}
\begin{frame}[fragile]
\frametitle{REST from Scratch}

\begin{center}
REST = Representational State Transfer
\end{center}

\begin{itemize}
\item 고유한 식별자(URI)를 갖는 resource 존재
\item Resource의 representation을 표준 interface(e.g. HTTP)로 통신
\item 제약 조건
\begin{itemize}
\item Client-server
\item Stateless
\item Cacheable
\item Layered system
\item Code on demand
\item Uniform interface
\end{itemize}
\item REST style architecture의 예: WWW
\end{itemize}

\end{frame}

\begin{frame}[fragile,allowframebreaks]
\frametitle{Restfulie Example}

\begin{itemize}
\item Restfulie: \url{http://restfulie.caelum.com.br/}
\[
\mbox{REST} + \mbox{HATEOAS} \Rightarrow \mbox{Restfulie framework}
\]
\item HATEOAS: Hypermedia as the Engine of Application State
\end{itemize}

Example Resources:
\begin{itemize}
    \item trip: \mbtt{http://TRIP/f/523a3f23}
    \item payment: URI unknown at compile time
\end{itemize}

\begin{block}<1->{Trip}
\lstset{basicstyle=\ttfamily\small,language=xml}
\begin{lstlisting}
<flight>
 <information> <from>Seoul</from> <to>Busan</to> 
 </information>
 <value>100</value>
 <link rel="payment" href="http://PAY/a/23342q4"/>
</flight>
\end{lstlisting}
\end{block}

\pagebreak

TRIP과 별도 사이트인 PAY를 통해 비용 지불.

\begin{block}<1->{Code example: payment}
\lstset{basicstyle=\ttfamily\small,language=java}
\begin{lstlisting}
flight = Client.at('http://TRIP/f/523a3f23').get();
confirmation = 
    flight.getLink("payment").patch(cardInfo, value);
\end{lstlisting}
\end{block}

Calender에 여행 일정 추가 후, calendar에서 일정 삭제하면 TRIP과 PAY에서도 
취소 처리가 되도록 할 수도 있다.

\begin{block}<1->{Code example: calendar integration}
\lstset{basicstyle=\ttfamily\small,language=java}
\begin{lstlisting}
me = Client.at('http://CAL').with(auth).get();
me.link("calendar").patch(flight.link("self"));
\end{lstlisting}
\end{block}

\end{frame}

% \begin{frame}
% \frametitle{REST from Scratch}
% 
% Guilherme Silveira, Caelum.
%  
% REST = Representational State Transfer
% 
% \begin{itemize}
% \item 시공을 통합하는 어플리케이션을 만들면 좋겠네.
%   \begin{itemize}
%   \item google/apple apps: 지도, 캘린더, ...
%   \item 리치 클라이언트: 자바스크립트를 실행 할 수 있는. 보안 문제를
%     해결해야
%   \item URI망?
%   \end{itemize}
% \item  WEB, HTML, XHTML, microformats, json, xforms, ... 뭐시 너무 많다.
% \item DSL: comlexity를 숨기기 위한 방법. 유지보수는 어떻게?
% \item Versioning: 서버 기술도 급격히 계속 발전하고, 브라우저도
%   다양하고, 브라우저도 계속 개선되니... 버전 관리 방안도 중요
% \end{itemize}
% 
% \end{frame}

\subsection{DSL}
\begin{frame}
\frametitle{Domain Specific Language}

DSL의 두 종류:
\begin{itemize}
\item Internal DSL: Lisp, Ruby, Scala  등의 언어를 확장한 것
    \begin{itemize}
    \item 루비 때문에 DSL이 buzz word가 된 듯.
    \item Lisp의 매크로, staged computation 등 그리 새롭지도 않은 것인데...
    \end{itemize}
\item External DSL: 유닉스 시스템의 각종 설정 파일 처럼 독자적인  syntax를 갖는 것
    \begin{itemize}
    \item Internal DSL이 아닌 것 전부
    \item XML은 noisy syntax가 문제
    \end{itemize}
\end{itemize}

\REFERENCE{\url{http://qconsf.com/sf2010/presentation/Software+Design+in+the+21st+Century}}

\end{frame}



% \begin{frame}
% \frametitle{References}
% \bibliography{references}
% \end{frame}

\end{document}

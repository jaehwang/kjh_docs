\begin{frame}
\frametitle{FUSE(Filesystem in USErspace)}

\begin{itemize}
\item 완전한 filesystem을 userspace에서 구현
\item Linux kernel 2.4, 2.6에서 실행
\end{itemize}

\begin{center}
\pgfimage[width=.6\textwidth]{fuse_structure.png}
\end{center}

\REFERENCE{\url{http://fuse.sourceforge.net/}}

\end{frame}

\begin{frame}[shrink]
\frametitle{FUSE+Subversion}

Example: Conifer Systems사의 Cascade

\begin{center}
\pgfimage[width=.6\textwidth]{cfs.png}
\end{center}

\begin{itemize}
\item Cache
  \begin{itemize}
  \item Repository 데이터 caching
  \item Working directory를 cascade filesystem으로 mount
  \item Cache를 복수의 working directory에서 공유
  \end{itemize}
\item FUSE 기반의 Cascade Filesystem
  \begin{itemize}
  \item Working directory를 cascade filesystem으로 mount
  \item Cache를 복수의 working directory에서 공유
  \end{itemize}
\item Operation
  \begin{itemize}
  \item Working directory에서 파일 내용을 읽거나 바꿀 때만 file data 전송 
  \item \mbtt{svn}을 대체하는 \mbtt{csc} command line interface 제공
  \end{itemize}
\end{itemize}

\REFERENCE{\url{http://www.conifersystems.com/cascade/}}

\end{frame}

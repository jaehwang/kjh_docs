\begin{frame}
\frametitle{Clojure-Java Interop: A better Java than Java}

\begin{itemize}
\item Simplicity
  \begin{itemize}
  \item Succinct syntax
  \item 놀라운 것은 Lisp 기반의 언어가 괄호 때문에 이상해 보인다는
    주장이 옳지 않다는 증거를 제시
  \item 실제로 괄호 수가 Java보다 Clojure가 적음
  \end{itemize}

\item Immutable data $\rightarrow$ uncontrolled mutation이 없음
  \begin{itemize}
  \item 이건 정말 좋은 점
  \item Concurrent programming에 유리한 조건이 됨
  \end{itemize}

\item Direct access to JVM: unboxed math
  \begin{itemize}
  \item 숫자 같은 primitive type data를 object 가 아닌 primitive
    type으로 바로 사용할 수 있음. (v1.3 이후)
  \end{itemize}

\item Meta data: 임의의 annotation을 할 수 있음.
%  \begin{itemize}
%  \item Javadoc하고는 비교가 안 되는 멋진 기능이라는데... 뭔지 정확히
%    몰라서...
%  \end{itemize}
\end{itemize}

\end{frame}
